\section{Method}
\subsection{Formulation for Task} 
In this paper, we use the same inputs and outputs as Attentive and DeepChpme while also adding a gene expression matrix for each cell line. Using the same formulation as Cheng \emph{et al.} the task is formulated as measuing the gene expression as either up (1) or down (0) regulated. First, per cell line, a GRN is precomputed which utilizes a matrix $E$ of size $S\times G$ where $S$ denotes the number of samples in the expression matrix while $G$ represents the number of genes that were recorded. \\\\
Hence, for a sample gene, two pieces of information are fed. The first is $H$ which is a graph describing gene-gene interactions for a partcular cell line. In the case of this paper $H$ was an adjacency list representation of the graph. Second, per gene, a matrix $X$ of size $M\times T$ was utilized where $M$ denotes the number of histone marks utilized while $T$ is the total number of bin positions taken into account around the TSS site of a gene. \\\\
Overall, for the training data of the GCN, we utilized $H$ and a $N\times M\times T$ sized matrix where $N$ is the number of gene samples. The output, accordingly, is a $N$-sized vector with either 0 or 1.

\subsection{Construction of GRNs}
The gene regulatory networks for this paper are built using the standard method of random forests. Utilizing the grnboost2 algorithm from the arboreto package on \url{pypi.org}, a regression task was defined for each gene in a cell line. \\\\
Let $E_i$ denote the row vector containing all samples for gene $i$ in the expression matrix. Then, specifically, for gene $i$, a random forest model $R_i$ was defined where, $L(R_i(E_{1:G\setminus i}), E_i)$ is minimized with $L$ denoting the mean square error. Once this task is completed, denote the set $\mathfrak{N}_i = \{imp(R_i, j) \mid j \in \{1:G\}\setminus i\}$. Here, $imp(R_i, j)$ refers to the feature importance of $j$ in random forest model $R_i$. Then the final graph $H$ is constructed with the neighbor list of a node $i$ being $\mathfrak{I}_i = \{j \mid imp(R_i, j) > \bar{\mathfrak{N}_i} + s(\mathfrak{N}_i)\}$ with $s(\mathfrak{N}_i)$ denoting the sample standard deviation. Hence, $H = \{\mathfrak{I}_i\}$, $\forall i$.